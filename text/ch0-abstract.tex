Considering the exponential growth of today’s industry and the wastewater results of its processes, it needs to have an optimal treatment system for such effluent waters to mitigate the environmental impact generated by its discharges and comply with the environmental regulatory standards that are progressively increasing their demand. This leads to the need to innovate in the control and management information systems responsible for treating these residual waters in search of improvement. This thesis proposes developing an intelligent system that uses the data from the process and predicts its behaviour to provide support in decision making related to the operation of the \ac{WWTP}. %To carry out the development of this system.
%a \ac{MLP} neural network with 2 hidden layers and 22 neurons each is implemented, together with process variable analysis, time-series decomposition, correlation and autocorrelation techniques; it is possible to predict the \ac{COD} at the input of the bioreactor with a one-day window and a mean absolute percentage error (MAPE) of 10.8\%\, which places this work between the adequate ranges proposed in the literature.
 
 In recent years, the implementation and development of electronic measurement devices, data storage systems, management information systems, and the increase in computational processing capabilities have ushered in industrial data availability, volume, and complexity. Transformation of data into information allows a better understanding, modelling, and optimization of industrial processes. This work aims to predict, with a 1-day time window, some key variables in an Industrial \ac{WWTP} that assist in the decision-making process regarding its operation. Three different approaches: \ac{FFNN} with auto-regression, \ac{LSTM} neural network and \ac{SVR} predict the \ac{COD} at discharge point \ac{COD}\textsubscript{D}, COD in the equalizer output \ac{COD}\textsubscript{EQ}, and \ac{MLVSS}. Afterward, three ensemble strategies combine the model's output to enhance the prediction; an Average Ensemble, a Fusion ensemble, and a Selection Ensemble. Results show a comparison between the approach's performance and the ensemble's proposals. Different single model approaches and ensemble models achieve appropriate \ac{MAPE} values in comparison with the state-of-the-art works.


%Ideally, this abstract and the one that you submitted along with your \emph{notice of intention to submit (NOI)}\footnote{\url{https://www.latrobe.edu.au/researchers/grs/hdr/candidature/forms-and-resources}} should be identical. However, slight variations are permissible. 
