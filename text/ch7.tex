This chapter highlights the key conclusions reached from this research. Then, it presents the main contributions of this this work and ends proposing potential future work of this investigation. 

\section{Conclusions}
%The selection and characterization of the most significant variables of the wastewater treatment process have been carried out satisfactorily. Three different approaches have been  using correlation analysis, auto correlations and decomposition of the time series. With these variables, an intelligent system based on artificial neural networks was developed to be capable of giving an adequate prediction of chemical oxygen demand, one of the most suitable variables to measure the level of pollutant load in the water and make decisions. 

Machine learning methods present an accurate performance among computational techniques in predicting variables in a highly complex and nonlinear process like wastewater treatment. In this study, Feed-forward Neural Network, Long short-term Memory Neural networks and Support Vector Regression are implemented and adjusted to predict Volatile Suspended Solids and Chemical Oxygen Demand in both discharge effluent and equalizer output. Each individual model exhibits its goodness and limitations. Assembling the outcomes of the model set allows enhancing the final prediction of the system, obtaining a more reliable and robust predictor overall that can support the decision-making process of the wastewater treatment plant operation. For single models \ac{MAPE} values of the training dataset range from 5.23\% up to 21.3\% and testing \ac{MAPE}s from 5.67\% up to 31.8\% in contrast with ensemble models training \ac{MAPE}s that goes from 3.99\% up to 22.4\% and testing \ac{MAPE}s from 5.68\% up to 18.9\%, prediction results within the range of the performance reported by the literature which validates the performance of the proposed system. 


\section{Main Contribution}

This thesis presents an intelligent system that combines different strategies to achieve industrial variables prediction, using WWTP as a case-of-study and predicting one day interval COD and VSS. The system forecast three key variables for the process, providing support in the decision-making task to the plant operator which will have valuable information of variables future values.

This thesis makes the following contributions in the prediction and Wastewater  treatment fields:
\begin{itemize}
    \item An intelligent system that uses state-of-art Machine Learning techniques such as Feed-forward Neural Network mixed with Time series Decomposition, Long Short-term Memory Neural Network, and Support Vector Regressor to predict Wastewater treatment variables that obtains accurate results measured by MAPE.
    \item Ensemble approaches capable of combining different single approaches outputs building a more reliable, robust and accurate intelligent system.  
\end{itemize}
 

\section{Future Work}
Aiming for improvements in the industrial wastewater treatment process, it is considered for future works:
\begin{itemize}
    \item Expand the prediction of the system to other key variables of the process.
    \item Test the intelligent system in other stages of the Wastewater treatment plant.
    \item Test the intelligent system with data from a different plant that can measure different variables.
\end{itemize}
Regard industrial time series prediction is worth for future research:
\begin{itemize}
    \item Expand the prediction of the system to new industrial fields.
    \item Increase the prediction window of the system.
    \item Explore echo state Networks as promising time series modelling strategy.
\end{itemize}

%Give an outlook regarding your expectations for the overall development of your chosen field of research. List topics that were not covered by your work. Identify and discuss potential avenues for follow-up research that you would consider worthwhile pursuing.