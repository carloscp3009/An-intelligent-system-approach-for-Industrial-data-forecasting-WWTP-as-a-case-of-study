%% This document created by Scientific Word (R) Version 3.0

\documentclass{book}%
\usepackage{graphicx}
\usepackage{amsmath}
\usepackage{amsfonts}
\usepackage{amssymb}%
\setcounter{MaxMatrixCols}{30}
%TCIDATA{OutputFilter=latex2.dll}
%TCIDATA{Version=5.00.0.2606}
%TCIDATA{CSTFile=LaTeX Book.cst}
%TCIDATA{Created=Thu Oct 17 23:10:40 2002}
%TCIDATA{LastRevised=Friday, June 03, 2005 10:57:02}
%TCIDATA{<META NAME="GraphicsSave" CONTENT="32">}
%TCIDATA{<META NAME="SaveForMode" CONTENT="1">}
%TCIDATA{BibliographyScheme=Manual}
%TCIDATA{<META NAME="DocumentShell" CONTENT="Books\Standard LaTeX Book">}
\newcommand\titulo[2]{\textcolor{#1}{\underline{#2}}}
\newcommand\capitulo[2]{\begin{center}
\colorbox{azulc}{\textcolor{white}{\large CAPÍTULO #1}}  {\large
\textcolor{azulc}{#2}}
\end{center}}
\newcommand\apleccion[3]{
\begin{center}
\docLink{#1}{\includegraphics{../../../../images/interfaz/left.gif}}\ \ \docLink{#2}{\includegraphics{../../../../images/interfaz/right.gif}}\ \ \docLink[_top]{#3}{\includegraphics{../../../../images/interfaz/inicio.gif}}
\end{center}}
\newtheorem{theorem}{Theorem}
\newtheorem{acknowledgement}[theorem]{Acknowledgement}
\newtheorem{algorithm}[theorem]{Algorithm}
\newtheorem{axiom}[theorem]{Axiom}
\newtheorem{case}[theorem]{Case}
\newtheorem{claim}[theorem]{Claim}
\newtheorem{conclusion}[theorem]{Conclusion}
\newtheorem{condition}[theorem]{Condition}
\newtheorem{conjecture}[theorem]{Conjecture}
\newtheorem{corollary}[theorem]{Corollary}
\newtheorem{criterion}[theorem]{Criterion}
\newtheorem{definition}[theorem]{Definition}
\newtheorem{example}[theorem]{Example}
\newtheorem{exercise}[theorem]{Exercise}
\newtheorem{lemma}[theorem]{Lemma}
\newtheorem{notation}[theorem]{Notation}
\newtheorem{problem}[theorem]{Problem}
\newtheorem{proposition}[theorem]{Proposition}
\newtheorem{remark}[theorem]{Remark}
\newtheorem{solution}[theorem]{Solution}
\newtheorem{summary}[theorem]{Summary}
\newenvironment{proof}[1][Proof]{\textbf{#1.} }{\ \rule{0.5em}{0.5em}}
\begin{document}

\begin{quotation}
%Ojaaaaaaaaaaaaaaaaaaaaaaaaaaaaaaaaaa
%inicio encabezado
{\color{gray}
\begin{tabular}
[c]{@{\extracolsep{\fill}}lcr}%
\docLink{../../leccion6/anova.tex}{\includegraphics{../../../../images/navegacion/anterior.gif}}
\begin{tabular}
[c]{l}%
{\color {darkgray} {\small Analisis de varianza}}\\
\\
\\
\end{tabular}
&
\docLink[_top]{../../../../index.html}{\includegraphics{../../../../images/navegacion/inicio.gif}}
\docLink{../../../../docs_curso/contenido.html}{\includegraphics{../../../../images/navegacion/contenido.gif}}
\docLink{../../../../docs_curso/descripcion.html}{\includegraphics{../../../../images/navegacion/descripcion.gif}}
\docLink[_top]{../../../../docs_curso/profesor.html}{\includegraphics{../../../../images/navegacion/profesor.gif}} &
\begin{tabular}
[c]{r}%
{\color {darkgray} {\small Prueba de significancia para cada coeficiente de regresion}}%
\\
\\
\\
\end{tabular}
\docLink{../leccion7-2/hipo-ind-beta.tex}{\includegraphics{../../../../images/navegacion/siguiente.gif}}\\
&  &
\end{tabular}
}
%fin encabezado
%Ojaaaaaaaaaaaaaaaaaaaaaaaaaaaaaaaaaaaa

\end{quotation}

\begin{center}
\includegraphics{../../../../images/line_col.gif} \bigskip
\textcolor{green}{\textbf{Prueba de significancia de la
regresión}} \bigskip\includegraphics{../../../../images/line_col.gif}
\end{center}

\begin{quotation}
La prueba de significancia de la regresi\'{o}n es una de la pruebas de
hip\'{o}tesis utilizadas para medir la bondad de ajuste del modelo. \ Esta
prueba determina si existe una relaci\'{o}n lineal entre la variable respuesta
$Y$ y alguna de las variables regresoras $X_{1},X_{2},\ldots,X_{n} $. \ La
hip\'{o}tesis estad\'{\i}stica adecuada es

\bigskip%

\begin{align*}
H_{0}  & :\beta_{1}=\beta_{1}=\cdots=\beta_{k}=0\\
H_{0}  & :\beta_{j}\neq0 \ \text{ para al menos un }j;\text{ \ }j=1,2,\ldots,k
\end{align*}


\bigskip

Al rechazar la hip\'{o}tesis nula se concluye que al menos una de las
variables regresoras contribuye significativamente al modelo.

\bigskip

La prueba estad\'{\i}stica utilizada es

\bigskip%

\begin{align*}
F  & =\dfrac{CM_{\text{Regresi\'{o}n}}}{CM_{\text{Error}}}\\
& =\dfrac{SC\left(  \text{Regresi\'{o}n}/b_{0}\right)  /\left(  p-\right)
1}{CM_{\text{Error}}}\\
& =\dfrac{\left(  \mathbf{b}^{\prime}\mathbf{X}^{\prime}\mathbf{Y-}%
n\overline{Y}^{2}\right)  /\left(  p-1\right)  }{CM_{\text{Error}}}%
\end{align*}


\bigskip

La cual asumiendo que la hip\'{o}tesis nula es cierta se distribuye $\ F$ con
$k$ grados de libertad en el numerador y $n-k-1$ grados de libertad en el denominador.

\bigskip

Se rechaza la hip\'{o}tesis nula si el valor calculado de la estad\'{\i}stica
de prueba es mayor que el valor te\'{o}rico de la distribuci\'{o}n $F\left(
\alpha,k,n-k-1\right)  $.

\bigskip\colorbox{green}{\textcolor{white}{ Ejemplo}} \bigskip

La hip\'{o}tesis es dada por%

\begin{align*}
H_{0}  & :\beta_{1}=\beta_{2}=0\\
H_{0}  & :\beta_{j}\neq0\text{ para al menos un }j\text{ \ }j=1,2
\end{align*}


La prueba estad\'{\i}stica utilizada es

\bigskip%

\begin{align*}
F  & =\dfrac{SC\left(  \text{Regresi\'{o}n}/b_{0}\right)  /\left(  p-1\right)
}{SC\left(  \text{Residual}\right)  /\left(  n-p\right)  }\\
& =\dfrac{2618,98/\left(  3-1\right)  }{43,1606/\left(  8-3\right)  }\\
& =151,70
\end{align*}


Luego como el valor P=0,00 entonces se rechaza la hip\'{o}tesis nula lo cual
significa que al menos una de las variables regresoras $X_{1}$ o $X_{2}$
contribuye significativamente al modelo.

%Ojoooooooooooooooooooooooooooooooo
\newline

{\color{gray}
\begin{tabular}
[c]{@{\extracolsep{\fill}}lcr}
&  & \\
\docLink{../../leccion6/anova.tex}{\includegraphics{../../../../images/navegacion/anterior.gif}}
\begin{tabular}
[c]{l}%
{\color {darkgray} {\small Analisis de varianza}}\\
\\
\\
\end{tabular}
&
\docLink[_top]{../../../../index.html}{\includegraphics{../../../../images/navegacion/inicio.gif}}
\docLink{../../../../docs_curso/contenido.html}{\includegraphics{../../../../images/navegacion/contenido.gif}}
\docLink{../../../../docs_curso/descripcion.html}{\includegraphics{../../../../images/navegacion/descripcion.gif}}
\docLink[_top]{../../../../docs_curso/profesor.html}{\includegraphics{../../../../images/navegacion/profesor.gif}} &
\begin{tabular}
[c]{r}%
{\color {darkgray} {\small Prueba de significancia para cada coeficiente de regresion}}%
\\
\\
\\
\end{tabular}
\docLink{../leccion7-2/hipo-ind-beta.tex}{\includegraphics{../../../../images/navegacion/siguiente.gif}}
\end{tabular}
}
\end{quotation}

\begin{flushright}
\docLink[_blank]{http://www.unal.edu.co}{\includegraphics{../../../../images/interfaz/copyright.gif}}%

\end{flushright}




\end{document}