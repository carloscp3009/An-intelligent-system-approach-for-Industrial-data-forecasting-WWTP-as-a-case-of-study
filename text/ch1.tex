\hinttext{This chapter provides an introduction to the work presented in this thesis. Specifically, the motivation in the research area, the pursued aims and the main contributions are briefly described. Finally, the chapter concludes with an overview of the structure and contents of the thesis.}

\section{Motivation}
\label{s:Motivation}


\section{Objectives}
\label{s:Objectives}
The research conducted in this dissertation aims to implement and experiment with different machine learning models to achieve a prediction of time series variables from an industrial field.


\textbf{General Objective:} Develop an intelligent system for industrial variables monitoring and forecasting.

\textbf{Specific Objectives:}

\begin{itemize}
\item Identify and characterize principal variables that carry out a proper forecasting.
\item Develop an intelligent system based on the key variables obtained and different forecasting models.
\item Evaluate each model performance to determine which model is suitable under different testing conditions.
\end{itemize}

\section{Thesis Question}
\label{s:Question}
The principal question addressed in this dissertation is:


\section{Contributions}
\label{s:Contributions}

Our key contributions include:

\begin{enumerate}

  \item \hinttext{List all important contributions that you intend to make.}
  
  \item \hinttext{Be brief!}
  
  \item \hinttext{The number of contributions does not have to match the number of contribution chapters.}

\end{enumerate}

\section{Thesis Outline}
\label{s:Outline}

The remainder of this thesis is organized as follows. \hinttext{Note: Depending on your research, you will have to change the structure of the thesis. For example, if you have many contribution chapters that are only loosely related, it might be useful to have a separate related works and/or experiments section per chapter.}

\begin{description}

  \item[Chapter \ref{c:Background}] provides background information relevant to the field of research. \hinttext{Provide details regarding topics and concepts that you will refer to throughout the thesis and that cannot be assumed to be already known by typical readers. A background chapter is not mandatory, but chances are that you need one. See \texttt{text/ch2.tex} for details.}
  
  \item[Chapter \ref{c:Related-Works}] provides a detailed discussion of the major existing contributions to the field and how they relate to each other and our proposed algorithm ... \hinttext{This must be adapted depending on your research. See \texttt{text/ch3.tex} for details.}
  
  \item[Chapter \ref{c:Contribution-1}] details the properties and development of ... \hinttext{Write at least 1-2 sentences more to give the reader an idea what your first major contribution chapter is about. See \texttt{text/ch4.tex} for details.}
  
  \item[Chapter \ref{c:Contribution-2}] details the properties and development of the ... algorithm. \hinttext{Write at least 1-2 sentences more to give readers an idea what your second major contribution chapter is about. See \texttt{text/ch5.tex} for details.}
  
  \hinttext{Expand this list if you have further contribution chapters.}
  
  \item[Chapter \ref{c:Experiments}] details testing of our proposed ... algorithm in an experimental setting. \hinttext{Write at least 1-2 sentences more to give an idea what the chapter is about. See \texttt{text/ch6.tex} for details.}
  
  \item[Chapter \ref{c:Conclusions}] concludes this thesis. First, we summarize our work. Then, we high-
  light interesting aspects that we did not fully cover and provide a thorough discussion of potential topics and areas, with respect to ..., that may be interesting for future research activities. \hinttext{A conclusion chapter is mandatory. Please do not copy. Rewrite this! See \texttt{text/ch7.tex} for details.}
  
\end{description}



