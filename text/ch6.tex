\hinttext{!!!ACTION REQUIRED!!!}
\hinttext{The structure I defined is generic and will most likely have to be adapted. I suggest that you skim through the pages and then clear the files \texttt{text/ch2.tex} to \texttt{text/ch7.tex} before you start writing.}

This is just an example. You may want to restructure this chapter, or even integrate it into you contribution chapter. 

\section{Experiment Setup}
\label{s:Experiment-Setup}

Define the goal of the experiment. If you compare algorithms, define the metrics/techniques that you will use to compare them and why.


\subsection{Environment and System Setup}
\label{s:Experiment-Env}

Detail the properties of the environmental conditions of your experiment. What hardware did you use? What compiler?


\subsection{Implemented Algorithms}
\label{s:Experiment-Algo}

\hinttext{Now you need to outline the scope of the experiment. For example, by listing the algorithms that you have implemented:}

In this section, we list the algorithms that we have included in our experimental study.

\begin{description} 
%  \item[Na\"{i}ve Space Folding] is the classic method that was originally proposed by \citeauthor{HerbertF:1965:Dune}~\cite{HerbertF:1965:Dune} to utilize the Holtzmann effect to fold space.
  
 % \item[Post-Butlerian Algorithm:] This approach uses the extended fold space theory as described by \citeauthor{HerbertB:2002:Butlerian-Jihad}~\cite{HerbertB:2002:Butlerian-Jihad}. It relies on infinite recursion and can currently only be implemented by beings with a heightened perception through massive infusion of \emph{Spice Melange}.

  \item[Ixian Inverse:] This is our new approach to fold space using an intelligent apparatus that implements the complete algorithm discussed in \autoref{s:Contribution-2-Major-2-Minor-2}.

\end{description}


\section{Results}
\label{s:Experiment-Results}

Use a somewhat expressive title for each experiment. This way you may easily reference them again in the summary.


\section{Summary}
\label{s:Experiments-Summary}

Summarize the key findings of the experiments you conducted.
